% This file stores LaTeX commands I found most useful in my daily 
% typesetting taskes so far.
% Author: Liangzhou Yi

% current version of this file includes content about how to  create 
% the following entities in LaTex:
%       mathmatic equations
%       trees
%       table
%       figures
%       code snippet

\documentclass[11pt]{article}

% packages goes here ...
\usepackage{fullpage}
\usepackage{amssymb, amsmath}       % this is for the mathmatics
\usepackage{color}                              % this is fo color used in text
\usepackage{listings}                           % this is for the code snippets
\usepackage{graphicx}                         % this is for the figures
\usepackage{qtree}                              % this is for the trees

\setlength{\parindent}{0pt} % set indention to zero for paragraph
\definecolor{dkgreen}{rgb}{0,0.6,0}   % self-defined color
\definecolor{gray}{rgb}{0.5,0.5,0.5}   % self-defined color

\title{A Not So Simple \LaTeX\ file}
\author{Liangzhou Yi}
\date{\today}

\begin{document}
\maketitle  % create the title

% This section is how we create mathmatic equations using LaTeX.
\section*{Mathmatics}
% we use ~\ref{label name} to reference a figure's number
Equation~\ref{eq:newton} is a displayed equation.\\ 
\begin{equation} \label{eq:newton} % define a label for current equation
f(x_{n+1}) = f(x_{n}) - \frac{f(x_{n})}{f\prime(x_{n})}
\end{equation}

We can also include inline formula (within text) here. 
$p(x) = \sum\limits_{i=1}^{n+1} y_{i}\varphi_{i}(x) = 
\sum\limits_{i=1}^{n+1} y_{i} (\prod\limits_{j \neq i}
\frac{x-x_{j}}{x_{i} - x_{j}})$.

% This section is how we create a tree using LaTeX and qtree package.
\section*{Tree}
\begin{figure}[h!]
\Tree [.sentence [.np [.art the ] [.n fair ] ] 
      [.vs answers ] 
      [.nps [.ns questions ] ] ] 
% when defining a tree, a space should be left before ]
\caption{The parse tree of a sentence using our modified grammar.}
\label{fig:unique}
\end{figure}

% This section is how we create a table using LaTex.
\section*{Table}
\begin{figure}[h!]
\centering
   \begin{tabular}{|l|l|l|l|}
        \hline
        ~         & Top-down & Bottom-up & Top-down Chart\\ \hline
        sentence: & 3      & 7      & 3  \\ 
        np :      & 5      & 6      & 3 \\
        \hline
    \end{tabular}
\caption{Comparision of three parsers.}
\label{fig:comp}
\end{figure}

Figure~\ref{fig:comp} is a table.

% This section is how we include a picture in LaTeX.
\section*{Figure}
Figure~\ref{fig:lagbasis} is a picture we input from a file.
\begin{figure}[h!]
  \centering
  \includegraphics[width=0.8\textwidth]%
    {lagbasis.eps}% picture filename
  \caption{Lagrange basis functions over $[-1,1]$.}
  \label{fig:lagbasis}
\end{figure}

% This section is how we include a code snippet in LaTeX.
\section*{Code snippet}

Refer to Figure~\ref{fig:vandermat} for the implementation of a method
producing a vandermone matrix given a row vector. \\*

% setting how code is shown below.
\lstset{frame=single, basicstyle=\footnotesize\ttfamily, 
commentstyle=\color{dkgreen},  breaklines = true, numbers=left,  
numberstyle=\color{gray}, keywordstyle=\color{blue}}

\begin{figure}[h!]
\centering
\lstinputlisting[language=Matlab]{vandermat.m}
\caption{Source code of method vandermat(x) in Matlab}
\label{fig:vandermat}
\end{figure}

\section*{Good references}
{\bf \LaTeX\ wikibooks}: http://en.wikibooks.org/wiki/LaTeX \\
{\bf The Not So Short Introduction to \LaTeX\ $2_{\varepsilon}$ } \\
{\bf \LaTeX\ to PDF}: 
    http://mintaka.sdsu.edu/GF/bibliog/latex/LaTeXtoPDF.html \\

\end{document}
